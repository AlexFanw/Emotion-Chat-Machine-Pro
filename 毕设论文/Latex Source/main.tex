\documentclass[supercite]{HustGraduPaper}

\title{基于情感的深度学习对话生成技术研究}
\author{范唯}
\school{计算机科学与技术}
\classnum{CS1703}
\stunum{U201714670}
\instructor{魏巍}
\date{2021年6月1日}

\usepackage{algorithm}
\usepackage{algpseudocode}
\usepackage{amsmath}
\usepackage{amsthm}
\usepackage{framed}
\usepackage{mathtools}
\usepackage{subcaption}
\usepackage{xltxtra}
\usepackage{bm}
\usepackage{tikz}
\usepackage{tikzscale}
\usepackage{pgfplots}

\pgfplotsset{compat=1.16}

\newcommand{\cfig}[3]{
  \begin{figure}[htb]
    \centering
    \includegraphics[width=#2\textwidth]{images/#1.tikz}
    \caption{#3}
    \label{fig:#1}
  \end{figure}
}
\newcommand{\sfig}[3]{
  \begin{subfigure}[b]{#2\textwidth}
    \includegraphics[width=\textwidth]{images/#1.tikz}
    \caption{#3}
    \label{fig:#1}
  \end{subfigure}
}
\newcommand{\xfig}[3]{
  \begin{figure}[htb]
    \centering
    #3
    \caption{#2}
    \label{fig:#1}
  \end{figure}
}

\newcommand{\rfig}[1]{\autoref{fig:#1}}
\newcommand{\ralg}[1]{\autoref{alg:#1}}
\newcommand{\rthm}[1]{\autoref{thm:#1}}
\newcommand{\rlem}[1]{\autoref{lem:#1}}
\newcommand{\reqn}[1]{\autoref{eqn:#1}}
\newcommand{\rtbl}[1]{\autoref{tbl:#1}}

\algnewcommand\Null{\textsc{null }}
\algnewcommand\algorithmicinput{\textbf{Input:}}
\algnewcommand\Input{\item[\algorithmicinput]}
\algnewcommand\algorithmicoutput{\textbf{Output:}}
\algnewcommand\Output{\item[\algorithmicoutput]}
\algnewcommand\algorithmicbreak{\textbf{break}}
\algnewcommand\Break{\algorithmicbreak}
\algnewcommand\algorithmiccontinue{\textbf{continue}}
\algnewcommand\Continue{\algorithmiccontinue}
\algnewcommand{\LeftCom}[1]{\State $\triangleright$ #1}

\newtheorem{thm}{定理}[section]
\newtheorem{lem}{引理}[section]

\colorlet{shadecolor}{black!15}

\theoremstyle{definition}
\newtheorem{alg}{算法}[section]

\def\thmautorefname~#1\null{定理~#1~\null}
\def\lemautorefname~#1\null{引理~#1~\null}
\def\algautorefname~#1\null{算法~#1~\null}

\begin{document}

\maketitle

\statement

\clearpage

\pagenumbering{Roman}

\begin{cnabstract}{情感对话生成;情感对话机器人;注意力机制;序列到序列模型;情感嵌入}

现如今,人工智能以及大数据从方方面面不断地影响着人们的生活。而许多智能AI助手已经可以非常准确的理解用户意图并给出贴切的回复。
然而这距离图灵的终极构想还相去甚远,机器能否表达出和人相匹配的智能依旧是目前的一大难题。因为人类在交流过程中,除了理性的信息
传输外,还有情感的交互夹杂其中,这类信息是隐式却至关重要的,所以想要进一步提升人机对话的沉浸感,则必须要在对话生成中加入情感
因素的考量。

新的情感对话系统在原先的基础上加入了情感自动选择模块,而不需要再进行人为的介入。整个系统从算法设计上来说,在序列到序列模型的基础上
加入了注意力机制、情感嵌入和语义嵌入。而从模块上来看,主要包括情感选择模块和回复生成模块。它不仅能准确回复用户的问题,还能在回复中
融入适当的情感。新系统的展示界面前端则采用Vue.js来设计,同时,后端服务则使用Django框架,这不仅能直观的展现对话系统功能,更提高了
交互的体验感。

新的系统不仅使回复具有多样性并与主题相关,避免统一而通用的回复,还使得生成的回复富有情感完全不依赖于手动标注。完美地解决了当下人机
对话系统中自动生成回复十分生硬的缺点,而用户体验也随之增强。


\end{cnabstract}

\begin{enabstract}{emotional dialogue generation, emotional chatbot, Attention Mechanisms, Sequence to Sequence, sentiment embedding}

Nowadays, artificial intelligence and big data continue to affect people's lives in all aspects. Artificial intelligence 
assistants can already understand users' intentions accurately and give appropriate responses.Nevertheless, this is still 
far from Turing's ultimate conception. Whether the machine can express the intelligence that matches the human is a big problem 
at present. Because in the process of human communication, besides logical information transmission, there are emotional 
interactions. This type of information is implicit but crucial. Therefore, if we want to further enhance the immersion 
of human-machine dialogue, we must consider emotion factor in dialogue generation.
  
The new emotional dialogue system adds an automatic emotional selection module on the original basis, without the need for human intervention. 
In terms of algorithm design, the entire system is based on sequence to sequence model with attention mechanism, emotional embedding 
and semantic embedding. From the perspective of modules, it mainly includes an emotion selection module and a response generation module. 
Not only can it accurately respond to the users' post, but also in the responses incorporate appropriate emotions. The front-end of the new system
is designed using Vue.js, while the back-end server uses the Django framework, which not only visually displays 
the functions of the dialogue system, but also improves interactive experience.

The new system makes the responses more diverse and relevant to the topic, avoiding uniform and universal replies, and also makes the 
generated responses full of emotion and does not rely on manual annotation at all. It perfectly solves the shortcoming of the current 
dialogue system that automatically generated responses are very blunt, and meanwhile, the user experience is also enhanced.



\end{enabstract}

\tableofcontents[level=2]
\clearpage

\pagenumbering{arabic}

\section{绪论}

在本章节中,首先会概述课题背景,然后展开描述国内外对于情感对话生成的研究现状,并且论述本课题的主要研究内容和方向。

\subsection{课题背景}
在自动对话系统日益完善的今天,人机对话已经在我们的生活中非常普及了,比如Siri、小度、小爱同学、天猫精灵等等,但是
业界中普及的一般是传统的问答模式,用户能很明显的感受到这是来自机器人的回复,简单直接且富有理性。但是,研究发现在
对话中加入情感因素,会大幅度提高整个对话的流畅程度和用户满意度。

目前现存的情感对话生成系统,大多是基于朱小燕、黄民烈老师团队开发的Emotion Chat Machine\cite{DBLP:journals/corr/ZhouHZZL17}
的一些变体,虽然能生成富有情感的回复,但是存在的问题是非常依赖于手动选择情感,在用户使用体验上较差。所以该类系统需要改进的地方是(1)生成的回复富有情感且完全
不依赖于手动标注;(2)回复具有多样性并与主题相关;(3)避免统一而通用的回复。基于此,对传统ECM模型进行优化改进有着重要的意义。

\subsection{国内外研究现状}

现有的对话生成模型主要是基于序列到序列架构(Seq2seq),这种架构可以有效的解决序列映射问题,且被广泛用于各大领域,其中主要包括机器翻译和图像标题。

首先在原生Seq2seq架构的研究工作中,Bahdanau等人使用双向长短期记忆网络(Bi-LSTM)的神经注意力机制去捕获句子中的重要语义信息,并且该架构能自动搜索上下文中的相关部分。
Luong等人则非常详尽地评估了具有不同注意力机制的序列到序列架构的性能。Jean等人提出了使用采样分类器的方法来解决巨量词汇引发的解码复杂度过高的问题。这些工作显著地提升了
Seq2seq模型的生成性能,并且加快了编/解码过程,为此后基于此架构的研究工作奠定了十分坚实的基础。

在近几年的研究中,科研人员还致力于研究如何将Seq2seq模型应用于对话系统,并且提出了不同的变体来解决不同特定领域的特定问题,例如主题感知模型、最大信息交互模型、层级循环模型、
增强型定向搜索模型等等,来产生更多元且更有信息含量的回复。并且随着研究的深入,逐渐将对话模型分为闲聊机器人与特定领域服务机器人。

在回复内容多样性方面,彭等人提出了基于变分自编码的情感对话模型(VAE-ECG),该架构利用来VAE的特性来更好更准确的对文本的潜在语义进行建模,使得生成的回复内容具有更好的拓展性。
刘等人则基于对话风格迁移的对话神经网络,提出了融合检索结果机制的生成对话模型,来避免重复单一的通用性回复。而诸多研究表面情感因素对于成功建立起拟人对话生成模型具有重要意义,Ghosh等人于2017年提出了情感语言模型,该模型可以根据指定的情感类别和情感强度来生成对应的结构化语言。Zhou
等人\cite{DBLP:journals/corr/ZhouHZZL17}所提出的基于条件和强化学习的自动编码框架,同样也是在给定的情感下生成最为合适的回复,刘等人提出了基于PAD情感状态模型的对话生成
框架,PAD简单直接的将情感纬度分为愉悦度、激活度和优势度,理论上可以使用这三个维度来表示所有的情感。但是上述的研究一直致力于提升生成回复的流利度和情感丰富度,但都需要人来手动
介入去选择注入最佳的情感类别,显然这是十分不理想的。所以本课题的主要研究方向则是在Zhou\cite{DBLP:journals/corr/ZhouHZZL17}等人的传统ECM框架下进行升级,加入自动情感选择模块。

\subsection{研究目的和主要内容}
目前来说,对话模型主要是使用序列到序列模型(Sequence to Sequence),序列到序列(Seq2seq)注意力模型通常是基于深度 RNN 的编码/解码器的体系结构。但是该模型在情感对话上存在着以下几个关键性问题:

(1)回答缺乏多样性,即容易产生通用回答,这是因为语料库中存在大量多对一的回答使得模型回答比较单一。

(2)缺少情感的编码和解码,即模型不考虑回答和问题之间的情感关系,只关心逻辑关系。

(3)需要手动选择最佳的情感类别,没有情感自动选择模块。

基于以上三个主要问题,本课题中的模型构建将主要聚焦于以下三点:

(1)语义嵌入(Semantic Embedding),基于注意力机制针对文本中的主题相关词进行建模,并在回复中围绕主题相关词语进行相似性扩展。

(2)情感嵌入(Sentiment Embedding),将情感理解成一种特殊的语义来进行编码,将其作为语义和主题的一个维度来进行嵌入,使生成的回复具有与主题联系更密切的情感倾向。

(3)最佳情感选择器(Optimal Emotion Selector),因为实际对话过程中不会有人实时来为回复进行情感选择,所以自动情感选择模块显得尤为重要。但值得注意的是情感选择
并不只是单纯的将post中的情感提取出来,然后在response中复用。因为不会有人想在生气的阐述观点时,听到对方也愤怒的回复他。

而且在之前的情感对话生成中,回复中所嵌入的情感都是单一的情感,但是人类对于情感的感知十分微妙,在言语中所包含的情感也不是只有一个种类,
所以在本课题中提出了情感向量空间并且假定情感概率的分布在整个向量空间上,从而使得生成的回复不会被单一的情绪所约束。此外,同样也希望能为
聊天机器人注入自我角色和情感偏好,也就是独有的性格,能让它在对话生成过程中有自己独立的个人身份和语言风格(比如积极、悲观、易怒、可爱甜美等),
而不是完全取决于用户的情感输入来产生对应的情绪,但这并不是此次课题的主要研究内容,可作为未来展望。




\subsection{论文结构}
本文的主要结构如下:

第一章中简明扼要地介绍了本课题的背景,情感对话生成模型的国内外研究现状,以及原有的模型架构存在的诸多问题,并提出了大致的优化改进方案。

第二章详细阐明了系统相关的技术背景,其中包含了词嵌入原理、序列到序列模型、注意力机制等算法模型的底层逻辑。

第三章介绍系统的整体结果设计,主要由情感选择模块与回复生成模块两大部分构成,并描述了展示界面的前后端设计逻辑。

第四章基于前面的设计部分进行具体实现,主要包括基于tensorflow的情感选择器和回复生成器,以及Vue.js+Django的交互展示界面实现过程。

第五章中描述了测试环境与基本方案,并横向对比了相似模型架构的测试结果,其中评估的主要内容包括语义流畅度与情感准确度。

第六章是对整个研究工作的总结与展望,总结了本系统相较于前人研究工作的优势,并展望未来能迭代升级的模块。


\section{技术背景概述} 
\subsection{词嵌入}
\subsection{序列到序列模型}
\subsection{注意力机制}
\subsection{本章小结}
这是一个算法。这是一个算法。这是一个算法。这是一个算法。这是一个算法。这是一个
算法。这是一个算法。这是一个算法。这是一个算法。这是一个算法。这是一个算法。这
是一个算法。这是一个算法。这是一个算法。这是一个算法。

\begin{shaded*}\begin{alg}{一个复杂算法}
  \label{alg:apb}
  \begin{algorithmic}
    \Input Two numbers $a$ and $b$
    \Output The sum of $a$ and $b$
    \Procedure{A-Plus-B}{$a, b$}
      \If $a = 0$
        \State \Return $b$
      \EndIf
      \State $res \gets 0$
      \While{$b \neq 0$}
        \State Increase $res$ by $1$
        \State $b \gets b - 1$
      \EndWhile
      \State \Return $res$
    \EndProcedure
  \end{algorithmic}
\end{alg}\end{shaded*}

\begin{thm} \label{thm:gsb-apprx}
  \ralg{apb}所示的算法是正确的。
\end{thm}
\begin{proof}[证明]
  显然,此处略去。
\end{proof}

我是字。我是字。我是字。我是字。我是字。我是字。我是字。我是字。我是字。我是字
。我是字。我是字。我是字。我是字。我是字。我是字。我是字。我是字。我是字。我是
字。我是字。我是字。我是字。我是字。我是字。我是字。我是字。我是字。


这是一张表。这是一张表。这是一张表。这是一张表。这是一张表。这是一张表。这是一
张表。这是一张表。这是一张表。这是一张表。这是一张表。这是一张表。这是一张表。
这是一张表。这是一张表。这是一张表。

\begin{generaltab}{这是一张表}{tbl:hmm}
  \begin{tabular}{c|ccc}
    \toprule
    我是字 & 第二列 & 第三列 & 第四列 \\
    \midrule
    第一行 & $1$ & $1$ & $4$ \\
    第二行 & $5$ & $1$ & $4$ \\
    \bottomrule
  \end{tabular}
\end{generaltab}

\rtbl{hmm}是有味道的。

我是字。我是字。我是字。我是字。我是字。我是字。我是字。我是字。我是字。我是字
。我是字。我是字。我是字。我是字。我是字。我是字。我是字。我是字。我是字。我是
字。我是字。我是字。我是字。我是字。我是字。我是字。我是字。我是字。


这是一堆表。这是一堆表。这是一堆表。这是一堆表。这是一堆表。这是一堆表。这是一
堆表。这是一堆表。这是一堆表。这是一堆表。这是一堆表。这是一堆表。这是一堆表。
这是一堆图。这是一堆图。这是一堆图。

\cfig{bpg-1}{0.8}{一张大图}

\xfig{bpg-l}{两张小图}{
  \sfig{bpg-la}{0.3}{小图}
  \sfig{bpg-lb}{0.3}{小图}
}

\rfig{bpg-1}很好看,\rfig{bpg-la}和\rfig{bpg-lb}因为缩得太小了不那么好看。

我是字。我是字。我是字。我是字。我是字。我是字。我是字。我是字。我是字。我是字
。我是字。我是字。我是字。我是字。我是字。我是字。我是字。我是字。我是字。我是
字。我是字。我是字。我是字。我是字。我是字。我是字。我是字。我是字。

\section{情感对话生成系统设计}
\subsection{系统结构设计}
\subsection{情感选择模块}
\subsection{回复生成模块}
\subsection{损失函数}
\subsection{本章小结}
\section{情感对话生成系统实现}
\subsection{情感选择器}
\subsection{回复生成器}
\subsection{本章小结}
\section{性能评估与分析}
\subsection{测试环境与方案}
\subsection{语义流畅度评估}
\subsection{情感准确度评估}
\subsection{本章小结}
\section{总结与展望}

\begin{thankpage}

感谢CCF给我这次机会。感谢CCTV给我这次机会。感谢HUST给我这次机会。感谢大萝卜给
我这次机会。感谢CCF给我这次机会。感谢CCTV给我这次机会。感谢HUST给我这次机会。
感谢大萝卜给我这次机会。

感谢CCF给我这次机会。感谢CCTV给我这次机会。感谢HUST给我这次机会。感谢大萝卜给
我这次机会。感谢CCF给我这次机会。感谢CCTV给我这次机会。感谢HUST给我这次机会。
感谢大萝卜给我这次机会。

这份模板根据\url{https://github.com/skinaze/HUSTPaperTemp}\cite{ski17}修改,添
加了一些CS人常用的东西。


\end{thankpage}

\nocite{*}

\bibliography{main}

\end{document}

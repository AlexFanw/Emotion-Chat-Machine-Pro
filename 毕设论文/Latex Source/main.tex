\documentclass[supercite]{HustGraduPaper}

\title{基于情感的深度学习对话生成技术研究}
\author{范唯}
\school{计算机科学与技术}
\classnum{CS1703}
\stunum{U201714670}
\instructor{魏巍}
\date{2021年6月1日}

\usepackage{algorithm}
\usepackage{algpseudocode}
\usepackage{amsmath}
\usepackage{amsthm}
\usepackage{framed}
\usepackage{mathtools}
\usepackage{subcaption}
\usepackage{xltxtra}
\usepackage{bm}
\usepackage{tikz}
\usepackage{tikzscale}
\usepackage{pgfplots}

\pgfplotsset{compat=1.16}

\newcommand{\cfig}[3]{
  \begin{figure}[htb]
    \centering
    \includegraphics[width=#2\textwidth]{images/#1.tikz}
    \caption{#3}
    \label{fig:#1}
  \end{figure}
}
\newcommand{\sfig}[3]{
  \begin{subfigure}[b]{#2\textwidth}
    \includegraphics[width=\textwidth]{images/#1.tikz}
    \caption{#3}
    \label{fig:#1}
  \end{subfigure}
}
\newcommand{\xfig}[3]{
  \begin{figure}[htb]
    \centering
    #3
    \caption{#2}
    \label{fig:#1}
  \end{figure}
}

\newcommand{\rfig}[1]{\autoref{fig:#1}}
\newcommand{\ralg}[1]{\autoref{alg:#1}}
\newcommand{\rthm}[1]{\autoref{thm:#1}}
\newcommand{\rlem}[1]{\autoref{lem:#1}}
\newcommand{\reqn}[1]{\autoref{eqn:#1}}
\newcommand{\rtbl}[1]{\autoref{tbl:#1}}

\algnewcommand\Null{\textsc{null }}
\algnewcommand\algorithmicinput{\textbf{Input:}}
\algnewcommand\Input{\item[\algorithmicinput]}
\algnewcommand\algorithmicoutput{\textbf{Output:}}
\algnewcommand\Output{\item[\algorithmicoutput]}
\algnewcommand\algorithmicbreak{\textbf{break}}
\algnewcommand\Break{\algorithmicbreak}
\algnewcommand\algorithmiccontinue{\textbf{continue}}
\algnewcommand\Continue{\algorithmiccontinue}
\algnewcommand{\LeftCom}[1]{\State $\triangleright$ #1}

\newtheorem{thm}{定理}[section]
\newtheorem{lem}{引理}[section]

\colorlet{shadecolor}{black!15}

\theoremstyle{definition}
\newtheorem{alg}{算法}[section]

\def\thmautorefname~#1\null{定理~#1~\null}
\def\lemautorefname~#1\null{引理~#1~\null}
\def\algautorefname~#1\null{算法~#1~\null}

\begin{document}

\maketitle

\statement

\clearpage

\pagenumbering{Roman}

\begin{cnabstract}{情感对话生成;情感对话机器人;注意力机制;序列到序列模型;情感嵌入}

现如今,人工智能以及大数据从方方面面不断地影响着人们的生活。而许多智能AI助手已经可以非常准确的理解用户意图并给出贴切的回复。
然而这距离图灵的终极构想还相去甚远,机器能否表达出和人相匹配的智能依旧是目前的一大难题。因为人类在交流过程中,除了理性的信息
传输外,还有情感的交互夹杂其中,这类信息是隐式却至关重要的,所以想要进一步提升人机对话的沉浸感,则必须要在对话生成中加入情感
因素的考量。

新的情感对话系统在原先的基础上加入了情感自动选择模块,而不需要再进行人为的介入。整个系统从算法设计上来说,在序列到序列模型的基础上
加入了注意力机制、情感嵌入和语义嵌入。而从模块上来看,主要包括情感选择模块和回复生成模块。它不仅能准确回复用户的问题,还能在回复中
融入适当的情感。新系统的展示界面前端则采用Vue.js来设计,同时,后端服务则使用Django框架,这不仅能直观的展现对话系统功能,更提高了
交互的体验感。

新的系统不仅使回复具有多样性并与主题相关,避免统一而通用的回复,还使得生成的回复富有情感完全不依赖于手动标注。完美地解决了当下人机
对话系统中自动生成回复十分生硬的缺点,而用户体验也随之增强。


\end{cnabstract}

\begin{enabstract}{emotional dialogue generation, emotional chatbot, Attention Mechanisms, Sequence to Sequence, sentiment embedding}

Nowadays, artificial intelligence and big data continue to affect people's lives in all aspects. Artificial intelligence 
assistants can already understand users' intentions accurately and give appropriate responses.Nevertheless, this is still 
far from Turing's ultimate conception. Whether the machine can express the intelligence that matches the human is a big problem 
at present. Because in the process of human communication, besides logical information transmission, there are emotional 
interactions. This type of information is implicit but crucial. Therefore, if we want to further enhance the immersion 
of human-machine dialogue, we must consider emotion factor in dialogue generation.
  
The new emotional dialogue system adds an automatic emotional selection module on the original basis, without the need for human intervention. 
In terms of algorithm design, the entire system is based on sequence to sequence model with attention mechanism, emotional embedding 
and semantic embedding. From the perspective of modules, it mainly includes an emotion selection module and a response generation module. 
Not only can it accurately respond to the users' post, but also in the responses incorporate appropriate emotions. The front-end of the new system
is designed using Vue.js, while the back-end server uses the Django framework, which not only visually displays 
the functions of the dialogue system, but also improves interactive experience.

The new system makes the responses more diverse and relevant to the topic, avoiding uniform and universal replies, and also makes the 
generated responses full of emotion and does not rely on manual annotation at all. It perfectly solves the shortcoming of the current 
dialogue system that automatically generated responses are very blunt, and meanwhile, the user experience is also enhanced.



\end{enabstract}

\tableofcontents[level=2]
\clearpage

\pagenumbering{arabic}

\section{绪论}

这份模板根据\url{https://github.com/skinaze/HUSTPaperTemp}\cite{ski17}修改,添
加了一些CS人常用的东西。

我是绪论。我是绪论。我是绪论。我是绪论。我是绪论。我是绪论。我是绪论。我是绪论
。我是绪论。我是绪论。我是绪论。我是绪论。我是绪论。我是绪论。我是绪论。我是绪
论。

\subsection{课题背景}

我是第一小节。我是第一小节。我是第一小节。我是第一小节。我是第一小节。我是第一
小节。我是第一小节。我是第一小节。我是第一小节。我是第一小节。我是第一小节。我
是第一小节。我是第一小节。我是第一小节。我是第一小节。我是第一小节。我是第一小
节。我是第一小节。

\subsection{国内外研究现状}

我是第二小节。我是第二小节。我是第二小节。我是第二小节。我是第二小节。我是第二
小节。我是第二小节。我是第二小节。我是第二小节。我是第二小节。我是第二小节。我
是第二小节。我是第二小节。我是第二小节。我是第二小节。我是第二小节。我是第二小
节。我是第二小节。

\subsection{研究内容和主要内容}

\subsection{论文结构}

\section{技术背景概述} 
\subsection{词嵌入}
\subsection{序列到序列模型}
\subsection{注意力机制}
\subsection{本章小结}
这是一个算法。这是一个算法。这是一个算法。这是一个算法。这是一个算法。这是一个
算法。这是一个算法。这是一个算法。这是一个算法。这是一个算法。这是一个算法。这
是一个算法。这是一个算法。这是一个算法。这是一个算法。

\begin{shaded*}\begin{alg}{一个复杂算法}
  \label{alg:apb}
  \begin{algorithmic}
    \Input Two numbers $a$ and $b$
    \Output The sum of $a$ and $b$
    \Procedure{A-Plus-B}{$a, b$}
      \If $a = 0$
        \State \Return $b$
      \EndIf
      \State $res \gets 0$
      \While{$b \neq 0$}
        \State Increase $res$ by $1$
        \State $b \gets b - 1$
      \EndWhile
      \State \Return $res$
    \EndProcedure
  \end{algorithmic}
\end{alg}\end{shaded*}

\begin{thm} \label{thm:gsb-apprx}
  \ralg{apb}所示的算法是正确的。
\end{thm}
\begin{proof}[证明]
  显然,此处略去。
\end{proof}

我是字。我是字。我是字。我是字。我是字。我是字。我是字。我是字。我是字。我是字
。我是字。我是字。我是字。我是字。我是字。我是字。我是字。我是字。我是字。我是
字。我是字。我是字。我是字。我是字。我是字。我是字。我是字。我是字。


这是一张表。这是一张表。这是一张表。这是一张表。这是一张表。这是一张表。这是一
张表。这是一张表。这是一张表。这是一张表。这是一张表。这是一张表。这是一张表。
这是一张表。这是一张表。这是一张表。

\begin{generaltab}{这是一张表}{tbl:hmm}
  \begin{tabular}{c|ccc}
    \toprule
    我是字 & 第二列 & 第三列 & 第四列 \\
    \midrule
    第一行 & $1$ & $1$ & $4$ \\
    第二行 & $5$ & $1$ & $4$ \\
    \bottomrule
  \end{tabular}
\end{generaltab}

\rtbl{hmm}是有味道的。

我是字。我是字。我是字。我是字。我是字。我是字。我是字。我是字。我是字。我是字
。我是字。我是字。我是字。我是字。我是字。我是字。我是字。我是字。我是字。我是
字。我是字。我是字。我是字。我是字。我是字。我是字。我是字。我是字。


这是一堆表。这是一堆表。这是一堆表。这是一堆表。这是一堆表。这是一堆表。这是一
堆表。这是一堆表。这是一堆表。这是一堆表。这是一堆表。这是一堆表。这是一堆表。
这是一堆图。这是一堆图。这是一堆图。

\cfig{bpg-1}{0.8}{一张大图}

\xfig{bpg-l}{两张小图}{
  \sfig{bpg-la}{0.3}{小图}
  \sfig{bpg-lb}{0.3}{小图}
}

\rfig{bpg-1}很好看,\rfig{bpg-la}和\rfig{bpg-lb}因为缩得太小了不那么好看。

我是字。我是字。我是字。我是字。我是字。我是字。我是字。我是字。我是字。我是字
。我是字。我是字。我是字。我是字。我是字。我是字。我是字。我是字。我是字。我是
字。我是字。我是字。我是字。我是字。我是字。我是字。我是字。我是字。

\section{情感对话生成系统设计}
\subsection{系统结构设计}
\subsection{情感选择器}
\subsection{回复生成器}
\subsection{损失函数}
\subsection{本章小结}
\section{情感对话生成系统实现}
\subsection{情感选择模块}
\subsection{回复生成模块}
\subsection{本章小结}
\section{性能评估与分析}
\subsection{测试环境与方案}
\subsection{语义流畅度评估}
\subsection{情感准确度评估}
\subsection{本章小结}
\section{总结与展望}

\begin{thankpage}

感谢CCF给我这次机会。感谢CCTV给我这次机会。感谢HUST给我这次机会。感谢大萝卜给
我这次机会。感谢CCF给我这次机会。感谢CCTV给我这次机会。感谢HUST给我这次机会。
感谢大萝卜给我这次机会。

感谢CCF给我这次机会。感谢CCTV给我这次机会。感谢HUST给我这次机会。感谢大萝卜给
我这次机会。感谢CCF给我这次机会。感谢CCTV给我这次机会。感谢HUST给我这次机会。
感谢大萝卜给我这次机会。

\end{thankpage}

\nocite{*}

\bibliography{sample}

\end{document}

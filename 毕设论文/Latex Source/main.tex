\documentclass[supercite]{HustGraduPaper}

\title{基于情感的深度学习对话生成}
\author{范唯}
\school{计算机科学与技术}
\classnum{CS1703}
\stunum{U201714670}
\instructor{魏巍}
\date{2021年6月1日}

\usepackage{algorithm}
\usepackage{algpseudocode}
\usepackage{amsmath}
\usepackage{amsthm}
\usepackage{framed}
\usepackage{mathtools}
\usepackage{subcaption}
\usepackage{xltxtra}
\usepackage{bm}
\usepackage{tikz}
\usepackage{tikzscale}
\usepackage{pgfplots}

\pgfplotsset{compat=1.16}

\newcommand{\cfig}[3]{
  \begin{figure}[htb]
    \centering
    \includegraphics[width=#2\textwidth]{images/#1.tikz}
    \caption{#3}
    \label{fig:#1}
  \end{figure}
}
\newcommand{\sfig}[3]{
  \begin{subfigure}[b]{#2\textwidth}
    \includegraphics[width=\textwidth]{images/#1.tikz}
    \caption{#3}
    \label{fig:#1}
  \end{subfigure}
}
\newcommand{\xfig}[3]{
  \begin{figure}[htb]
    \centering
    #3
    \caption{#2}
    \label{fig:#1}
  \end{figure}
}

\newcommand{\rfig}[1]{\autoref{fig:#1}}
\newcommand{\ralg}[1]{\autoref{alg:#1}}
\newcommand{\rthm}[1]{\autoref{thm:#1}}
\newcommand{\rlem}[1]{\autoref{lem:#1}}
\newcommand{\reqn}[1]{\autoref{eqn:#1}}
\newcommand{\rtbl}[1]{\autoref{tbl:#1}}

\algnewcommand\Null{\textsc{null }}
\algnewcommand\algorithmicinput{\textbf{Input:}}
\algnewcommand\Input{\item[\algorithmicinput]}
\algnewcommand\algorithmicoutput{\textbf{Output:}}
\algnewcommand\Output{\item[\algorithmicoutput]}
\algnewcommand\algorithmicbreak{\textbf{break}}
\algnewcommand\Break{\algorithmicbreak}
\algnewcommand\algorithmiccontinue{\textbf{continue}}
\algnewcommand\Continue{\algorithmiccontinue}
\algnewcommand{\LeftCom}[1]{\State $\triangleright$ #1}

\newtheorem{thm}{定理}[section]
\newtheorem{lem}{引理}[section]

\colorlet{shadecolor}{black!15}

\theoremstyle{definition}
\newtheorem{alg}{算法}[section]

\def\thmautorefname~#1\null{定理~#1~\null}
\def\lemautorefname~#1\null{引理~#1~\null}
\def\algautorefname~#1\null{算法~#1~\null}

\begin{document}

\maketitle

\statement

\clearpage

\pagenumbering{Roman}

\begin{cnabstract}{中文关键词;中文关键词;中文关键词}

中文摘要。中文摘要。中文摘要。中文摘要。中文摘要。中文摘要。中文摘要。中文摘要
。中文摘要。中文摘要。中文摘要。中文摘要。中文摘要。中文摘要。中文摘要。中文摘
要。中文摘要。中文摘要。中文摘要。中文摘要。中文摘要。中文摘要。中文摘要。中文
摘要。中文摘要。中文摘要。中文摘要。中文摘要。中文摘要。中文摘要。中文摘要。

中文摘要。中文摘要。中文摘要。中文摘要。中文摘要。中文摘要。中文摘要。中文摘要
。中文摘要。中文摘要。中文摘要。中文摘要。中文摘要。中文摘要。中文摘要。中文摘
要。中文摘要。中文摘要。中文摘要。中文摘要。中文摘要。中文摘要。中文摘要。中文
摘要。中文摘要。中文摘要。中文摘要。中文摘要。中文摘要。中文摘要。中文摘要。

\end{cnabstract}

\begin{enabstract}{keyword in English, keyword in English, keyword in English}

Abstract in English. Abstract in English. Abstract in English. Abstract in
English. Abstract in English. Abstract in English. Abstract in English.
Abstract in English. Abstract in English. Abstract in English. Abstract in
English. Abstract in English. Abstract in English. Abstract in English.

Abstract in English. Abstract in English. Abstract in English. Abstract in
English. Abstract in English. Abstract in English. Abstract in English.
Abstract in English. Abstract in English. Abstract in English. Abstract in
English. Abstract in English. Abstract in English. Abstract in English.

\end{enabstract}

\tableofcontents[level=2]
\clearpage

\pagenumbering{arabic}

\section{绪论}

这份模板根据\url{https://github.com/skinaze/HUSTPaperTemp}\cite{ski17}修改,添
加了一些CS人常用的东西。

我是绪论。我是绪论。我是绪论。我是绪论。我是绪论。我是绪论。我是绪论。我是绪论
。我是绪论。我是绪论。我是绪论。我是绪论。我是绪论。我是绪论。我是绪论。我是绪
论。

\subsection{第一小节}

我是第一小节。我是第一小节。我是第一小节。我是第一小节。我是第一小节。我是第一
小节。我是第一小节。我是第一小节。我是第一小节。我是第一小节。我是第一小节。我
是第一小节。我是第一小节。我是第一小节。我是第一小节。我是第一小节。我是第一小
节。我是第一小节。

\subsection{第二小节}

我是第二小节。我是第二小节。我是第二小节。我是第二小节。我是第二小节。我是第二
小节。我是第二小节。我是第二小节。我是第二小节。我是第二小节。我是第二小节。我
是第二小节。我是第二小节。我是第二小节。我是第二小节。我是第二小节。我是第二小
节。我是第二小节。

\section{还可以写很多}

\subsection{一个算法}

这是一个算法。这是一个算法。这是一个算法。这是一个算法。这是一个算法。这是一个
算法。这是一个算法。这是一个算法。这是一个算法。这是一个算法。这是一个算法。这
是一个算法。这是一个算法。这是一个算法。这是一个算法。

\begin{shaded*}\begin{alg}{一个复杂算法}
  \label{alg:apb}
  \begin{algorithmic}
    \Input Two numbers $a$ and $b$
    \Output The sum of $a$ and $b$
    \Procedure{A-Plus-B}{$a, b$}
      \If $a = 0$
        \State \Return $b$
      \EndIf
      \State $res \gets 0$
      \While{$b \neq 0$}
        \State Increase $res$ by $1$
        \State $b \gets b - 1$
      \EndWhile
      \State \Return $res$
    \EndProcedure
  \end{algorithmic}
\end{alg}\end{shaded*}

\begin{thm} \label{thm:gsb-apprx}
  \ralg{apb}所示的算法是正确的。
\end{thm}
\begin{proof}[证明]
  显然,此处略去。
\end{proof}

我是字。我是字。我是字。我是字。我是字。我是字。我是字。我是字。我是字。我是字
。我是字。我是字。我是字。我是字。我是字。我是字。我是字。我是字。我是字。我是
字。我是字。我是字。我是字。我是字。我是字。我是字。我是字。我是字。

\subsection{一张表}

这是一张表。这是一张表。这是一张表。这是一张表。这是一张表。这是一张表。这是一
张表。这是一张表。这是一张表。这是一张表。这是一张表。这是一张表。这是一张表。
这是一张表。这是一张表。这是一张表。

\begin{generaltab}{这是一张表}{tbl:hmm}
  \begin{tabular}{c|ccc}
    \toprule
    我是字 & 第二列 & 第三列 & 第四列 \\
    \midrule
    第一行 & $1$ & $1$ & $4$ \\
    第二行 & $5$ & $1$ & $4$ \\
    \bottomrule
  \end{tabular}
\end{generaltab}

\rtbl{hmm}是有味道的。

我是字。我是字。我是字。我是字。我是字。我是字。我是字。我是字。我是字。我是字
。我是字。我是字。我是字。我是字。我是字。我是字。我是字。我是字。我是字。我是
字。我是字。我是字。我是字。我是字。我是字。我是字。我是字。我是字。

\subsection{一堆表}

这是一堆表。这是一堆表。这是一堆表。这是一堆表。这是一堆表。这是一堆表。这是一
堆表。这是一堆表。这是一堆表。这是一堆表。这是一堆表。这是一堆表。这是一堆表。
这是一堆图。这是一堆图。这是一堆图。

\cfig{bpg-1}{0.8}{一张大图}

\xfig{bpg-l}{两张小图}{
  \sfig{bpg-la}{0.3}{小图}
  \sfig{bpg-lb}{0.3}{小图}
}

\rfig{bpg-1}很好看,\rfig{bpg-la}和\rfig{bpg-lb}因为缩得太小了不那么好看。

我是字。我是字。我是字。我是字。我是字。我是字。我是字。我是字。我是字。我是字
。我是字。我是字。我是字。我是字。我是字。我是字。我是字。我是字。我是字。我是
字。我是字。我是字。我是字。我是字。我是字。我是字。我是字。我是字。

\begin{thankpage}

感谢CCF给我这次机会。感谢CCTV给我这次机会。感谢HUST给我这次机会。感谢大萝卜给
我这次机会。感谢CCF给我这次机会。感谢CCTV给我这次机会。感谢HUST给我这次机会。
感谢大萝卜给我这次机会。

感谢CCF给我这次机会。感谢CCTV给我这次机会。感谢HUST给我这次机会。感谢大萝卜给
我这次机会。感谢CCF给我这次机会。感谢CCTV给我这次机会。感谢HUST给我这次机会。
感谢大萝卜给我这次机会。

\end{thankpage}

\nocite{*}

\bibliography{sample}

\end{document}
